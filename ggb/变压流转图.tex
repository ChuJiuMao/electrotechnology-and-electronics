\documentclass[tikz,convert,convert={outext=.pdf,command=\unexpanded{}}]{standalone}

\usepackage{array}
\usepackage{amsmath}
\usepackage{graphicx}
\usepackage{tikz}
\usepackage{tikz-3dplot}
\usepackage{ctex}
\usepackage{circledtext}
\usepackage{pgfplots}
\usepackage[american]{circuitikz}
\pgfplotsset{compat=1.15}
\usepgfplotslibrary{polar}


\newcommand \lightblue {cyan!20}
\newcommand \lightred {red!20}

\tdplotsetmaincoords{75}{105}
%视角旋转命令:\tdplotsetmaincoords{num}{num}
%含义:{绕x轴顺时针旋转<num>度}{绕y轴顺时针旋转<num>度}
%普通的右手系设定为{60}{120},即绕x轴顺时针旋转60°,绕y轴顺时针旋转120°

\begin{document}
    \centering
    \begin{tikzpicture}[
        scale=0.7,
        >=latex
        ]

    \draw [->]node[left]{$u_1$}(0,0)--(2,0)node[right]{$i_1$};
    \draw [->](2.7,0)--(4.7,0)node[right]{$\varPhi _{M1}$};
    \draw [->](2.35,-0.35)--(2.35,-2.35)node[below]{$\varPhi _{\sigma 1}$};
    \draw [->](5.9,0)--(7.9,0)node[right]{$\varPhi$};
    \draw [->](8.2,-0.35)--(8.2,-2.35)node[below]{$e_2$};
    \draw [->](8.6,-2.7)--(10.2,-2.7)node[right]{$i_2$};
    \draw (8.2,0.35)--(8.2,2.35);
    \draw [->](8.2,2.35)--(5.9,2.35)node[left]{$e_1$};
    \draw (4.9,2.35)--(2.35,2.35);
    \draw [->](2.35,2.35)--(2.35,0.35);

    \draw [->](10.6,-2.4)--(10.6,-0.4)node[above]{$\varPhi _{M2}$};
    \draw [->](9.9,0)--(8.5,0);

    \draw [->](11,-2.7)--(13,-2.7)node[right]{$\varPhi _{\sigma 2}$};




    \end{tikzpicture}
\end{document}