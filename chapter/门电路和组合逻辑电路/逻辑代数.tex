\section{\K 逻辑代数}
\Par 我们先给出基础的9条规则
\begin{equation*}
    \begin{matrix}
        \begin{aligned}
        \boldsymbol{A}\cdot 1&=\boldsymbol{A}\\
        \boldsymbol{A}\cdot 0&=0\\
        \boldsymbol{A}\cdot \boldsymbol{A}&=\boldsymbol{A}\\
        \boldsymbol{A}\cdot \overline{\boldsymbol{A}}&=0\\
    \end{aligned}&		\begin{aligned}
        \boldsymbol{A}+1&=1\\
        \boldsymbol{A}+0&=\boldsymbol{A}\\
        \boldsymbol{A}+\boldsymbol{A}&=\boldsymbol{A}\\
        \boldsymbol{A}+\overline{\boldsymbol{A}}&=1\\
    \end{aligned}&		\begin{array}{c}
        \overline{\overline{\boldsymbol{A}}}=\boldsymbol{A}\\
        \\
        \\
        \\
    \end{array}\\
    \end{matrix}
\end{equation*}

然后需要指出,不同于普通运算,逻辑代数的与运算和或运算具有同等优先级,这体现在交换律

\textbf{交换律}
\begin{align}
	\boldsymbol{A}\cdot \left( \boldsymbol{B}+\boldsymbol{C} \right) &=\left( \boldsymbol{A}\cdot \boldsymbol{B} \right) +\left( \boldsymbol{A}\cdot \boldsymbol{C} \right)\\
	\boldsymbol{A}+\left( \boldsymbol{B}\cdot \boldsymbol{C} \right) &=\left( \boldsymbol{A}+\boldsymbol{B} \right) \cdot \left( \boldsymbol{A}+\boldsymbol{C} \right)
\end{align}

\textbf{结合律}
\begin{align}
	\boldsymbol{A}\cdot \boldsymbol{B}\cdot \boldsymbol{C}&=\left( \boldsymbol{A}\cdot \boldsymbol{B} \right) \cdot \boldsymbol{C}=\boldsymbol{A}\cdot \left( \boldsymbol{B}\cdot \boldsymbol{C} \right)\\
	\boldsymbol{A}+\boldsymbol{B}+\boldsymbol{C}&=\left( \boldsymbol{A}+\boldsymbol{B} \right) +\boldsymbol{C}=\boldsymbol{A}+\left( \boldsymbol{B}+\boldsymbol{C} \right)
\end{align}

\textbf{交换律}
\begin{align}
	\boldsymbol{A}+\boldsymbol{B}&=\boldsymbol{B}+\boldsymbol{A}\\
	\boldsymbol{A}\cdot \boldsymbol{B}&=\boldsymbol{B}\cdot \boldsymbol{A}
\end{align}

\textbf{反演律}
\begin{align}
	\overline{\boldsymbol{A}\cdot \boldsymbol{B}}&=\overline{\boldsymbol{A}}+\overline{\boldsymbol{B}}\\
	\overline{\boldsymbol{A}+\boldsymbol{B}}&=\overline{\boldsymbol{A}}\cdot \overline{\boldsymbol{B}}
\end{align}

另有一些常用的公式:

\textbf{合并律}