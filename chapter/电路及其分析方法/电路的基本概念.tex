\section{\K 电路的基本概念}
\begin{equation*}
    \text{电路的作用}\begin{cases}
        \text{能量的传输、分流与转换}\\
        \text{信息的传递、转换与处理}\\
    \end{cases}
\end{equation*}

\Par 电路中电源或信号源的电压或电流称为\textbf{激励},它推动电路工作.由于激励,电路各部分产生的电压和电流被称作\textbf{响应}.\textbf{所谓的激励和响应都是电路中的参量,例如电动势、电压、电流,而不是具体的电子元件}.

\hl{参考方向}:在复杂电路的计算中,我们往往不能事先确定电路中各参量的方向,因此我们在计算前往往会规定一个\textbf{参考方向}便于计算,算出来的结果符号为正,就说明参考方向与实际方向相符;算出来的结果符号为负,就说明参考方向与实际方向相反.

\Par 同样为了方便计算,我们规定\textbf{电压的方向}:由高电势指向低电势;\textbf{电动势方向}:在\textbf{电源内部}由低电势指向高电势.

\hl{关联方向}:当电压方向与电流方向相同时,我们称它们之间的关系是\textbf{关联方向};当电压方向与电流方向相反时,我们称它们之间的关系是\textbf{非关联方向}.它们之间关联与否决定了计算用到的公式,例如欧姆定律在\textbf{关联方向}中应当为
\begin{equation}
    U=IR
\end{equation}
而在\textbf{非关联方向}中,应当为
\begin{equation}
    U=-IR
\end{equation}

\textbf{外特性曲线}:电源端电压U与输出电流之间关系的曲线,被称作\textbf{电源的外特性曲线}.\textbf{注意},仅有电源有外特性曲线,普通电路上叫做伏安特性曲线.

\hl{电功关系}:我们知道电功率公式为
\begin{equation}
    P=UI
\end{equation}
事实上在关联方向和非关联方向状态下的电功率公式是相同的,此时得到的结果的正负反映了该元件是吸收电流携带的能量,还是让电流增加了能量.比如,在电源内部电动势E的方向与电流I方向相反,这表示电源让电流增加了能量;再比如,电灯泡处,电流方向与电压方向相同,这表示小灯泡吸收了电流携带的能量.

\Par 综上所述,我们知道了:\textbf{计算结果电功为正表示消耗电路中的能量,电功为负表示提供电路中的能量}.

\begin{table}[htbp]
    \centering
    \caption{各概念中符号的意义}
    \begin{tblr}{row{odd} = {azure8}, 
        row{even} = {gray8},
        colspec={ll},
        row{1} = {c,2em,azure2,fg=white}
        }
        范围 & 正负号含义 \\
        参考方向 & 参考方向与实际方向是否相符\\
        关联方向 & 决定了计算要用的公式\\
        电功关系 & 反映该元件对电功的吸收与放出\\
        KVL方程  &  绕行方向与参考方向之间的关系\\
    \end{tblr}
\end{table}