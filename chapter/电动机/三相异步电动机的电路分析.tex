\section{\K 三相异步电动机的电路分析}
\subsection{\K 定子电路}
根据上文分析
\begin{equation*}
    u_1+e_{\sigma 1}+e_1=R_1i_1
\end{equation*}
将$e_{\sigma 1},R_1i_1$忽略,则有
\begin{equation*}
    u_1\approx -e_1
\end{equation*}
同时有
\begin{equation*}
    U_1\approx E_1=4.44f_1N_1\varPhi 
\end{equation*}
其中,$f_1$为电流变换频率,$N_1$为一次绕组的匝数.
\begin{equation*}
    f_1=\frac{n_0}{60}\cdot p
\end{equation*}
其中,$n_0$为磁场转速.

\subsection{\K 转子电路}
\begin{enumerate}
    \item [\circledtext{1}]转子频率
    
    转子频率为
    \begin{equation*}
        f_2=\frac{n_0-n}{60}\cdot p
    \end{equation*}
    $n_0-n$为磁场转速与转子转速之差,表示两者的相对转速,亦即磁场的切割速度\footnote{有没有感觉和我们之前提到的转速差$s$很像}
    \begin{equation*}
        f_2=\frac{n_0-n}{60}\cdot p=\frac{n_0-n}{n_0}\cdot \frac{n_0p}{60}=sf_1
    \end{equation*}
    在电机起动\footnote{为什么不是“启动”?}时,$n=0$,$s=1$,$f_2=f_1$,此时
    \begin{equation*}
        E_{20}=4.44f_2N_2\varPhi =4.44f_1N_2\varPhi 
    \end{equation*}
    而一般情况下,转子的感应电动势应为
    \begin{equation*}
        e_2=E_2=4.44f_2N_2\varPhi =4.44sf_1N_2\varPhi 
    \end{equation*}
    也由此可以得出
    \begin{equation*}
        E_2=sE_{20}
    \end{equation*}

    \item [\circledtext{2}]转子感抗
    
    对于转子的感抗,我们根据定义有
    \begin{equation*}
        X_{L2}=\omega _2L_2=2\pi f_2L_2
    \end{equation*}
    代入$f_2=sf_1$,可以得到
    \begin{equation*}
        X_{L2}=2\pi sf_1L_2
    \end{equation*} 
    电机起动时
    \begin{equation*}
        X_{L20}=2\pi f_1L_2
    \end{equation*} 
    从而有
    \begin{equation*}
        X_{L2}=sX_{L20}
    \end{equation*} 
    \item[\circledtext{3}]电流
    
    \begin{equation*}
        I_2=\frac{E_2}{\left| Z_2 \right|}=\frac{E_2}{\sqrt{R_{2}^{2}+X_{2}^{2}}}=\frac{sE_{20}}{\sqrt{R_{2}^{2}+\left( sX_{20} \right) ^2}}
    \end{equation*}
    \item[\circledtext{4}]功率因数  
    
    \begin{equation*}
        \cos \varphi =\frac{R}{\left| Z_2 \right|}=\frac{R}{\sqrt{R_{2}^{2}+\left( sX_{20} \right) ^2}}
    \end{equation*}


\end{enumerate}

